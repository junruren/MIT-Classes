\documentclass[10pt,landscape]{article}

\usepackage[margin=0.5in]{geometry}
\usepackage{amsmath, amssymb, bm}
\usepackage{multicol}
\usepackage{enumitem}
\usepackage{xcolor}

\usepackage{listings}
\usepackage{xcolor}

\lstset{
  language=Matlab,
  basicstyle=\ttfamily\footnotesize, % smaller font; try \scriptsize or \tiny
  keywordstyle=\color{blue},
  commentstyle=\color{teal},
  stringstyle=\color{red},
  numbers=left,
  numberstyle=\tiny,
  stepnumber=1,
  numbersep=5pt,
  frame=single,          % draws a box around the code
  columns=fullflexible,  % tighter spacing
  showstringspaces=false,
  breaklines=true,
  xleftmargin=0pt,
  xrightmargin=0pt,
}




\setlength{\parskip}{2pt}
\setlength{\parindent}{0pt}
\setlength{\columnsep}{0.3in}

\newcommand{\vect}[1]{\mathbf{#1}}
\newcommand{\R}{\mathbb{R}}
\newcommand{\defn}[1]{\textcolor{blue}{#1}}
\newcommand{\ex}[1]{\textcolor{red}{#1}}

\author{AAG}
\date{December 16, 2025}

\begin{document}
\small
{\large 10.34 Final Exam Formula Sheet \hfill December 16, 2025\par}
{\large V0.1 \hfill AAG\par}
\begin{center}
\end{center}

\begin{multicols}{3}


\noindent\hrule
\vspace{2pt}

%%%%%%%%%%%%%%%%%%%%%%%%%%%%%%%%%%%%%%%%%%%%%%%%%%%%
\section*{1. Numerical error \& complexity}
\noindent\hrule
\vspace{1pt}
%%%%%%%%%%%%%%%%%%%%%%%%%%%%%%%%%%%%%%%%%%%%%%%%%%%%

\defn{Floating point:} real numbers stored with finite precision; roundoff errors inevitable.

\defn{Machine epsilon:} smallest \(\varepsilon>0\) such that \(1+\varepsilon\neq 1\) in floating point.

Floating point and error.
\begin{itemize}[leftmargin=*]
\item Double precision: \(\varepsilon_{\text{mach}} \approx 2.2\times 10^{-16}\).
\item Relative error: \(e_{\text{rel}} = (\hat{x}-x)/x\); absolute error: \(e_{\text{abs}} = \hat{x}-x\).
\item Avoid subtracting nearly equal numbers (catastrophic cancellation).
\end{itemize}

\defn{Truncation error:} error from approximating an infinite process (e.g.\ Taylor series, step size).

\defn{Big-O:} \(f(n)=O(g(n))\) if \(|f(n)|\le C|g(n)|\) for large \(n\).

Big-O complexity.
\begin{itemize}[leftmargin=*]
\item Dot product: \(O(n)\).
\item Matrix--vector: \(O(mn)\).
\item Matrix--matrix: \(O(mnp)\).
\item Dense linear solve: \(O(n^3)\).
\end{itemize}

\ex{Example (catastrophic cancellation):}
\begin{equation*}
\text{Compute } 1-\cos(x)\ \text{for small }x.
\end{equation*}
Naive: \(1-\cos(x)\) loses precision. Better:
\begin{equation*}
1-\cos(x) = 2\sin^2(x/2),
\end{equation*}
which is numerically more stable for small \(x\).

\noindent\hrule
\vspace{2pt}

%%%%%%%%%%%%%%%%%%%%%%%%%%%%%%%%%%%%%%%%%%%%%%%%%%%%
\section*{2. Linear algebra essentials}
\noindent\hrule
\vspace{1pt}
%%%%%%%%%%%%%%%%%%%%%%%%%%%%%%%%%%%%%%%%%%%%%%%%%%%%

\defn{Norm:} function \(\|\cdot\|\) measuring vector or matrix ``size'', satisfying positivity, homogeneity, and the triangle inequality.

\defn{Condition number:} \(\kappa(A)=\|A\|\|A^{-1}\|\); measures sensitivity of the solution of \(Ax=b\) to perturbations.

\textbf{Norms \& conditioning.}
\begin{itemize}[leftmargin=*]
\item Vector \(p\)-norm: \(\|\vect{x}\|_p = \left(\sum_i |x_i|^p\right)^{1/p}\).
\item Matrix induced norms:
\(\|A\|_1 = \max_j \sum_i |a_{ij}|\),
\(\|A\|_\infty = \max_i \sum_j |a_{ij}|\).
\end{itemize}

\textbf{Some properties}
\begin{itemize}[leftmargin=*]
\item Outer product of two non-zero vectors $uv^T$ is always rank 1.
\item Product of the eigenvalues is the determinant. 
\item If columns (or rows) are nearly linearly dependent or have wildly different scales, the matrix is likely ill‑conditioned.
\item If the Hessian at a critical point has all eigenvalues positive (symmetric case), the Hessian is positive definite and the point is a strict local minimum.
\end{itemize}

\textbf{Singular vs nonsingular matrices.}

\defn{Nonsingular (invertible) \(A\in\R^{n\times n}\):}
\begin{itemize}[leftmargin=*]
\item \(\det(A)\neq 0\).
\item Inverse exists: \(\exists\,A^{-1}\) with \(AA^{-1}=A^{-1}A=I\).
\item Columns (and rows) are linearly independent.
\item Rank \(=n\); null space \(\mathcal{N}(A)=\{\vect{0}\}\).
\item For every \(\vect{b}\), the system \(A\vect{x}=\vect{b}\) has a unique solution \(\vect{x}=A^{-1}\vect{b}\).
\item \(0\) is not an eigenvalue of \(A\).
\end{itemize}

\defn{Singular \(A\in\R^{n\times n}\):}
\begin{itemize}[leftmargin=*]
\item \(\det(A)=0\).
\item No inverse exists.
\item Columns (or rows) linearly dependent; rank \(<n\).
\item Nontrivial null space: \(\exists\,\vect{x}\neq \vect{0}\) with \(A\vect{x}=\vect{0}\).
\item For some \(\vect{b}\), \(A\vect{x}=\vect{b}\) has no solution; for others, infinitely many.
\item \(0\) is an eigenvalue: \(\exists\,\vect{v}\neq \vect{0}\) with \(A\vect{v}=\vect{0}\).
\end{itemize}

\ex{Example (check singularity and implications):}
\begin{equation*}
A=
\begin{bmatrix}
1 & 2\\
2 & 4
\end{bmatrix},\quad
\det(A)=1\cdot 4-2\cdot 2=0\Rightarrow A \text{ is singular}.
\end{equation*}
So \(A^{-1}\) does not exist, columns are multiples, and there are nonzero solutions to \(A\vect{x}=\vect{0}\).

\textbf{Special matrices (examples and properties).}

\defn{Identity matrix:}
\begin{equation*}
I_3 =
\begin{bmatrix}
1 & 0 & 0\\
0 & 1 & 0\\
0 & 0 & 1
\end{bmatrix},\quad
AI_3=I_3A=A.
\end{equation*}
Eigenvalues are all \(1\); \(\det(I_3)=1\).

\defn{Diagonal matrix:}
\begin{equation*}
D =
\begin{bmatrix}
d_1 & 0 & 0\\
0 & d_2 & 0\\
0 & 0 & d_3
\end{bmatrix}.
\end{equation*}
\begin{itemize}[leftmargin=*]
\item \(\det(D)=d_1 d_2 d_3\).
\item Eigenvalues are \(\lambda_1=d_1,\ \lambda_2=d_2,\ \lambda_3=d_3\).
\item \(D^{-1}=\mathrm{diag}(1/d_1,1/d_2,1/d_3)\) if all \(d_i\neq 0\).
\end{itemize}

\defn{Upper triangular matrix:}
\begin{equation*}
U =
\begin{bmatrix}
u_{11} & u_{12} & u_{13}\\
0 & u_{22} & u_{23}\\
0 & 0 & u_{33}
\end{bmatrix}.
\end{equation*}
\begin{itemize}[leftmargin=*]
\item \(\det(U)=u_{11}u_{22}u_{33}\).
\item Eigenvalues are the diagonal entries: \(\lambda_i=u_{ii}\).
\item Solve \(U\vect{x}=\vect{b}\) by back-substitution in \(O(n^2)\).
\end{itemize}

\defn{Lower triangular matrix:}
\begin{equation*}
L =
\begin{bmatrix}
\ell_{11} & 0 & 0\\
\ell_{21} & \ell_{22} & 0\\
\ell_{31} & \ell_{32} & \ell_{33}
\end{bmatrix}.
\end{equation*}
\begin{itemize}[leftmargin=*]
\item \(\det(L)=\ell_{11}\ell_{22}\ell_{33}\).
\item Eigenvalues are \(\ell_{ii}\).
\item Solve \(L\vect{x}=\vect{b}\) by forward substitution.
\end{itemize}

\defn{Symmetric matrix:} \(A=A^T\); eigenvalues are real and eigenvectors can be chosen orthonormal.

\textbf{Four fundamental subspaces and solvability.}
\begin{itemize}[leftmargin=*]
\item Column space \(\mathcal{R}(A)\), null space \(\mathcal{N}(A)\), row space \(\mathcal{R}(A^T)\),
left null space \(\mathcal{N}(A^T)\).
\item Existence: \(A\vect{x}=\vect{b}\) solvable iff \(\vect{b}\in\mathcal{R}(A)\).
\item Uniqueness: solution unique iff \(\mathcal{N}(A)=\{\vect{0}\}\).
\end{itemize}

\textbf{Eigenvalues, eigenvectors, and algorithm sketch.}

\defn{Eigenpair:} \((\lambda,\vect{v})\) with \(\vect{v}\neq \vect{0}\) such that \(A\vect{v}=\lambda\vect{v}\).

Basic properties:
\begin{itemize}[leftmargin=*]
\item \(\det(A-\lambda I)=0\) gives the \defn{characteristic polynomial}; its roots are eigenvalues.
\item For each eigenvalue \(\lambda\), eigenvectors solve \((A-\lambda I)\vect{v}=\vect{0}\).
\item If \(A\) has a basis of eigenvectors, then \(A=V\Lambda V^{-1}\) with \(V=[\vect{v}_1\,\dots\,\vect{v}_n]\),
\(\Lambda=\mathrm{diag}(\lambda_1,\dots,\lambda_n)\).
\end{itemize}

\defn{Power method (sketch) for dominant eigenvalue:}
\begin{enumerate}[leftmargin=*]
\item Choose nonzero \(\vect{x}_0\).
\item For \(k=0,1,\dots\): compute \(\vect{y}_{k+1}=A\vect{x}_k\), then normalize \(\vect{x}_{k+1}=\vect{y}_{k+1}/\|\vect{y}_{k+1}\|\).
\item Rayleigh quotient \(\lambda_k = \vect{x}_k^T A \vect{x}_k\) approximates largest \(|\lambda|\).
\end{enumerate}

\defn{Practical eigenvalue algorithms:} QR iteration with shifts (used by \verb|eig|); repeated orthogonal similarity transforms drive \(A\) toward upper triangular (Schur form) whose diagonal entries approximate eigenvalues.

\ex{Example (eigenvalues and eigenvectors of a \(2\times 2\) matrix).}

Let
\begin{equation*}
A=
\begin{bmatrix}
2 & 1\\
1 & 2
\end{bmatrix}.
\end{equation*}

1) Eigenvalues: solve \(\det(A-\lambda I)=0\).
\begin{equation*}
A-\lambda I=
\begin{bmatrix}
2-\lambda & 1\\
1 & 2-\lambda
\end{bmatrix},\quad
\det(A-\lambda I)=(2-\lambda)^2-1.
\end{equation*}
Set this to zero:
\begin{equation*}
(2-\lambda)^2-1=0
\;\Rightarrow\;
2-\lambda=\pm 1
\;\Rightarrow\;
\lambda_1=3,\ \lambda_2=1.
\end{equation*}

2) Eigenvector for \(\lambda_1=3\): solve \((A-3I)\vect{v}_1=\vect{0}\).

We have
\[
A-3I=
\begin{bmatrix}
-1 & 1\\
1 & -1
\end{bmatrix},
\]
so the equation \((A-3I)\vect{v}_1=\vect{0}\) is
\[
-v_{1,1}+v_{1,2}=0 \;\Rightarrow\; v_{1,1}=v_{1,2}.
\]
Choose \(v_{1,1}=1\), so
\[
\vect{v}_1=
\begin{bmatrix}
1\\
1
\end{bmatrix},
\]
and the normalized eigenvector is
\[
\hat{\vect{v}}_1 = \frac{1}{\sqrt{2}}
\begin{bmatrix}
1\\
1
\end{bmatrix}.
\]

3) Eigenvector for \(\lambda_2=1\): solve \((A-I)\vect{v}_2=\vect{0}\).

We have
\[
A-I=
\begin{bmatrix}
1 & 1\\
1 & 1
\end{bmatrix},
\]
so
\[
v_{2,1}+v_{2,2}=0 \;\Rightarrow\; v_{2,2}=-v_{2,1}.
\]
Choose \(v_{2,1}=1\), so
\[
\vect{v}_2=
\begin{bmatrix}
1\\
-1
\end{bmatrix},
\]
and the normalized eigenvector is
\[
\hat{\vect{v}}_2 = \frac{1}{\sqrt{2}}
\begin{bmatrix}
1\\
-1
\end{bmatrix}.
\]

Check notation:
\[
A\hat{\vect{v}}_1 = \lambda_1 \hat{\vect{v}}_1,\quad
A\hat{\vect{v}}_2 = \lambda_2 \hat{\vect{v}}_2.
\]
With \(V=[\hat{\vect{v}}_1\ \hat{\vect{v}}_2]\) and \(\Lambda=\mathrm{diag}(\lambda_1,\lambda_2)\), we have
\[
A = V\Lambda V^{-1},
\]
and since \(A\) is symmetric, \(V\) can be chosen orthogonal so that \(V^{-1}=V^T\).


\subsection*{Orthogonal matrices and projectors}

\textbf{Orthogonal matrix.} $Q\in\mathbb{R}^{n\times n}$ is orthogonal if
\[
Q^T Q = QQ^T = I.
\]
Properties:
\[
Q^{-1} = Q^T,\qquad \|Qx\|_2 = \|x\|_2,\qquad (Qx)\cdot(Qy) = x\cdot y,
\]
\[
\det(Q) = \pm 1,\qquad |\lambda_i(Q)| = 1\ \text{for all eigenvalues } \lambda_i.
\]

\textbf{Orthogonal projector.} If $Q\in\mathbb{R}^{n\times k}$ has orthonormal columns, then
\[
P = QQ^T
\]
is the projector onto $\mathcal{R}(Q)$, with
\[
P^T = P,\quad P^2 = P,\quad \lambda_i(P)\in\{0,1\},\quad \operatorname{rank}(P)=k.
\]
For least squares, with design matrix $X$,
\[
\hat y = X\hat\theta = X(X^T X)^{-1}X^T y = P_X y.
\]

\subsection*{Positive (semi)definite symmetric matrices}

$A\in\mathbb{R}^{n\times n}$ symmetric.

\textbf{Positive definite (PD):}
\[
x^T A x > 0\ \ \forall x\neq 0.
\]

\textbf{Positive semidefinite (PSD):}
\[
x^T A x \ge 0\ \ \forall x.
\]

Equivalent facts for symmetric $A$:
\[
A\succ 0 \iff \lambda_i(A) > 0\ \forall i,\qquad
A\succeq 0 \iff \lambda_i(A) \ge 0\ \forall i.
\]

\textbf{Cholesky:} $A\succ 0 \iff \exists R\text{ upper triangular with }A = R^T R.$

\textbf{Optimization link.} If $H(x)$ (Hessian) is PD at a critical point $x^\star$:
\[
\nabla f(x^\star)=0,\ H(x^\star)\succ 0 \ \Rightarrow\ x^\star\ \text{is a strict local minimum (and locally unique).}
\]

\subsection*{Symmetric matrices and eigendecomposition}

For symmetric $A\in\mathbb{R}^{n\times n}$, there exists an orthogonal $W$ and real diagonal $\Lambda$ such that
\[
A = W\Lambda W^T,\quad W^T W = I,\quad \Lambda = \operatorname{diag}(\lambda_1,\dots,\lambda_n),\ \lambda_i\in\mathbb{R}.
\]

Useful identities:
\[
\det(A) = \prod_{i=1}^n \lambda_i,\qquad
\operatorname{tr}(A) = \sum_{i=1}^n \lambda_i.
\]

If an eigenvalue $\lambda$ has algebraic multiplicity $m$ and $A$ is symmetric, then its geometric multiplicity is also $m$, so
\[
\dim\mathcal{N}(A-\lambda I) = m,\qquad
\operatorname{rank}(A-\lambda I) = n - m.
\]

\subsection*{Rank--nullity and condition number}

For $A\in\mathbb{R}^{m\times n}$,
\[
\dim\mathcal{R}(A) + \dim\mathcal{N}(A) = n.
\]

If $A$ is invertible, the (matrix) condition number in the $2$-norm is
\[
\kappa_2(A) = \|A\|_2 \|A^{-1}\|_2 = \frac{\max_i |\lambda_i(A)|}{\min_i |\lambda_i(A)|}.
\]


\noindent %\hrule
\vspace{2pt}

%%%%%%%%%%%%%%%%%%%%%%%%%%%%%%%%%%%%%%%%%%%%%%%%%%%%
\section*{3. SNE and optimization}
\noindent\hrule
\vspace{1pt}
%%%%%%%%%%%%%%%%%%%%%%%%%%%%%%%%%%%%%%%%%%%%%%%%%%%%

\defn{Jacobian:} matrix of first partial derivatives:
\(J_{ij}=\partial f_i/\partial x_j\).

\defn{Newton's method (SNE):} iteratively solve linearized system to update guess.

Solve \(\vect{f}(\vect{x})=\vect{0}\):
\begin{equation*}
J(\vect{x}_k)\Delta \vect{x}_k = -\vect{f}(\vect{x}_k),\quad
\vect{x}_{k+1} = \vect{x}_k + \Delta \vect{x}_k.
\end{equation*}

\ex{Example (scalar Newton):}
Solve \(x^2-2=0\).
\begin{equation*}
f(x)=x^2-2,\quad f'(x)=2x,
\end{equation*}
\begin{equation*}
x_{k+1}=x_k-\frac{f(x_k)}{f'(x_k)}=x_k-\frac{x_k^2-2}{2x_k}.
\end{equation*}

\defn{Gradient:} vector of first partial derivatives; direction of steepest ascent.

\defn{Hessian:} matrix of second partial derivatives; local curvature.

Gradient/Hessian examples.

Quadratic:
\begin{equation*}
f(\vect{x}) = \tfrac12 \vect{x}^T Q \vect{x} + \vect{c}^T\vect{x}+d,\quad Q=Q^T,
\end{equation*}
\begin{equation*}
\nabla f(\vect{x}) = Q\vect{x} + \vect{c},\quad H(\vect{x}) = Q.
\end{equation*}

Least squares \(f(\vect{x}) = \tfrac12\|A\vect{x}-\vect{b}\|_2^2\):
\begin{equation*}
\nabla f(\vect{x}) = A^T(A\vect{x}-\vect{b}),\quad H(\vect{x}) = A^TA.
\end{equation*}

\ex{Example (minimize quadratic):}
Solve \(\min f(x)=\tfrac12 x^TQx+c^Tx\).
Set gradient \(Qx+c=0\Rightarrow x^*=-Q^{-1}c\), assuming \(Q\succ 0\).

\defn{Lagrangian:} \(\mathcal{L}(x,\lambda)=f(x)+\lambda^T c(x)\) for equality constraints \(c(x)=0\).

Equality-constrained optimum:
\begin{equation*}
\nabla_x \mathcal{L}=\nabla f + J_c^T\lambda=0,\quad c(x)=0.
\end{equation*}

\defn{Convex set:} $C\subset \mathbb{R}^n$ is convex if
\[
x_1,x_2\in C,\ t\in[0,1]\ \Rightarrow\ tx_1+(1-t)x_2\in C.
\]

\defn{Convex function:} $f:\mathbb{R}^n\to\mathbb{R}$ is convex if
\[
f(tx_1+(1-t)x_2)\le tf(x_1)+(1-t)f(x_2),\ \forall x_1,x_2,\ t\in[0,1].
\]
If $f$ is twice differentiable and $\nabla^2 f(x)\succeq 0$ for all $x$, then $f$ is convex. If $\nabla^2 f(x)\succ 0$ for all $x$, $f$ is \emph{strictly convex} and has at most one minimizer.

\defn{Unconstrained first-/second-order conditions:}
For $\min_x f(x)$ with $f$ differentiable:
\[
\nabla f(x^\star)=0\quad\text{(first-order necessary)}.
\]
If additionally $f$ is twice differentiable,
\[
\nabla f(x^\star)=0,\ \ \nabla^2 f(x^\star)\succ 0
\]
is a \emph{sufficient} condition for $x^\star$ to be a strict local minimizer (unique locally).

For convex $f$, any $x^\star$ with $\nabla f(x^\star)=0$ is a \emph{global} minimizer.

\defn{Equality-constrained KKT (first order):}
\[
\min_x f(x)\ \ \text{s.t.}\ \ c(x)=0,\ c:\mathbb{R}^n\to\mathbb{R}^m.
\]
Lagrangian
\[
\mathcal{L}(x,\lambda)=f(x)+\lambda c(x)^T.
\]
KKT system (necessary):
\[
\nabla_x\mathcal{L}(x^\star,\lambda^\star)=\nabla f(x^\star)+J_c(x^\star)^T\lambda^\star=0,\quad
c(x^\star)=0.
\]

\defn{KKT Newton system (matrix form):}
At iterate $(x_k,\lambda_k)$, one Newton step for the KKT conditions solves
\[
\begin{bmatrix}
\nabla^2_{xx}\mathcal{L}(x_k,\lambda_k) & J_c(x_k)^T \\
J_c(x_k) & 0
\end{bmatrix}
\begin{bmatrix}
\Delta x_k\\[2pt]\Delta\lambda_k
\end{bmatrix}
=
-
\begin{bmatrix}
\nabla f(x_k)+J_c(x_k)^T\lambda_k\\[2pt]
c(x_k)
\end{bmatrix},
\]
then updates $x_{k+1}=x_k+\Delta x_k,\ \lambda_{k+1}=\lambda_k+\Delta\lambda_k.$

\defn{Steepest descent (gradient descent):}
For unconstrained $\min f(x)$,
\[
d_k = -\nabla f(x_k),\quad x_{k+1}=x_k+\alpha_k d_k.
\]
- Fixed small $\alpha_k=\alpha$ $\Rightarrow$ guaranteed descent for sufficiently small $\alpha$, but slow convergence near optimum.
- Exact/line-search $\alpha_k$ (1D minimization along $d_k$) $\Rightarrow$ better convergence but more work per step.

\defn{Newton's method (optimization):}
\[
d_k = -\nabla^2 f(x_k)^{-1}\nabla f(x_k),\quad x_{k+1}=x_k+d_k.
\]
- Quadratic convergence near a solution if $\nabla^2 f(x^\star)$ is nonsingular (typically PD).
- May diverge or go uphill if $\nabla^2 f$ is indefinite (nonconvex problems).

\defn{Quasi-Newton / BFGS:}
Maintain an approximation $H_k\approx \nabla^2 f(x_k)^{-1}$ updated from gradients only.
\[
s_k = x_{k+1}-x_k,\quad y_k = \nabla f(x_{k+1})-\nabla f(x_k),
\]
\[
H_{k+1} =
\left(I-\frac{s_k y_k^T}{y_k^T s_k}\right)
H_k
\left(I-\frac{y_k s_k^T}{y_k^T s_k}\right)
+
\frac{s_k s_k^T}{y_k^T s_k}.
\]
Search direction:
\[
d_k = -H_k\nabla f(x_k).
\]
- Start with $H_0=I$ (first step = steepest descent).
- If $y_k^Ts_k>0$, BFGS preserves positive definiteness of $H_k$ and typically gives superlinear convergence.

\defn{Interior-point (barrier) method for inequality constraints:}
For
\[
\min_x f(x)\ \ \text{s.t.}\ \ h_i(x)\le 0,\ i=1,\dots,m,
\]
introduce barrier parameter $\mu>0$ and barrier function
\[
\phi_\mu(x) = f(x) - \mu\sum_{i=1}^m \log(h_i(x)),
\]
defined only for strictly feasible $x$ with $h_i(x)<0$.
Algorithm idea:
\begin{itemize}
\item For large $\mu$, approximately minimize $\phi_\mu(x)$ (e.g., with Newton).
\item Decrease $\mu\to 0$ and warm-start each new solve from the previous minimizer.
\end{itemize}
As $\mu\to 0$, solutions of the barrier problems approach a KKT point of the original constrained problem.




\noindent%\hrule
\vspace{2pt}

\pagebreak


%%%%%%%%%%%%%%%%%%%%%%%%%%%%%%%%%%%%%%%%%%%%%%%%%%%%
\section*{4. Probability basics}
\noindent\hrule
\vspace{1pt}
%%%%%%%%%%%%%%%%%%%%%%%%%%%%%%%%%%%%%%%%%%%%%%%%%%%%

\defn{Random variable (RV):} numerical outcome of a random experiment.

\defn{Discrete vs continuous RVs:}
\begin{itemize}[leftmargin=*]
\item Discrete: $X$ takes values in a countable set $\{x_k\}$, $P(X=x_k)=p_X(x_k)$, $\sum_k p_X(x_k)=1$.
\item Continuous: $X$ has density $f_X(x)\ge 0$, $P(a\le X\le b)=\int_a^b f_X(x)\,dx$, $\int_{-\infty}^{\infty} f_X(x)\,dx=1$.
\end{itemize}

\defn{PMF, PDF, CDF (definitions + intuition):}
\begin{itemize}[leftmargin=*]
\item \textbf{PMF (probability mass function)} for discrete $X$:
  \[
  p_X(k)=P(X=k).
  \]
  Properties: $p_X(k)\ge 0$ for all $k$; $\sum_k p_X(k)=1$. Intuition: a \emph{bar chart of exact probabilities} at each possible value. 

\item \textbf{PDF (probability density function)} for continuous $X$:
  \[
  f_X(x)\ge 0,\quad \int_{-\infty}^{\infty} f_X(x)\,dx=1.
  \]
  Probability over an interval: $P(a\le X\le b)=\int_a^b f_X(x)\,dx$. Properties: nonnegative; integrates to 1; the value $f_X(x)$ is a \emph{density}, not a probability, so $P(X=x)=0$. Intuition: a \emph{smooth curve of probability intensity}; area under the curve over a region is probability. 

\item \textbf{CDF (cumulative distribution function)} for any $X$:
  \[
  F_X(x)=P(X\le x).
  \]
  Discrete: $F_X(x)=\sum_{x_k\le x} p_X(x_k)$. Continuous: $F_X(x)=\int_{-\infty}^x f_X(t)\,dt$. Key properties: $0\le F_X(x)\le 1$; nondecreasing in $x$; $\lim_{x\to -\infty}F_X(x)=0$, $\lim_{x\to +\infty}F_X(x)=1$; right-continuous. Intuition: an \emph{accumulated probability curve}, showing how much probability mass has been accumulated up to threshold $x$.
\end{itemize}

\defn{Expectation:} mean; $\mathbb{E}[X]$.

Key formulas.
\begin{itemize}[leftmargin=*]
\item $\mathbb{E}[g(X)]=\sum_k g(k)p_X(k)$ or $\int g(x)f_X(x)\,dx$.
\item $\mathrm{Var}(X)=\mathbb{E}[X^2]-(\mathbb{E}[X])^2$.
\item $P(A|B)=P(A\cap B)/P(B)$.
\item Bayes: $P(A|B)=\dfrac{P(B|A)P(A)}{P(B)}$.
\end{itemize}

\defn{Marginal and conditional:}
Joint pmf/pdf $p_{X,Y}(x,y)$:
\[
p_X(x)=\sum_y p_{X,Y}(x,y)\quad\text{or}\quad p_X(x)=\int p_{X,Y}(x,y)\,dy,
\]
\[
p_{Y|X}(y|x)=\frac{p_{X,Y}(x,y)}{p_X(x)}.
\]

\defn{Convolution (sum of independent RVs):}
If $Z=X+Y$ and $X,Y$ independent continuous,
\[
f_Z(z)=\int_{-\infty}^\infty f_X(x)f_Y(z-x)\,dx.
\]
For discrete,
\[
p_Z(k)=\sum_i p_X(i)\,p_Y(k-i).
\]

\defn{Covariance and covariance matrix:}
\[
\mathrm{Cov}(X,Y)=\mathbb{E}[(X-\mathbb{E}X)(Y-\mathbb{E}Y)].
\]
For random vector $x\in\mathbb{R}^n$,
\[
C=\mathbb{E}\big[(x-\mu)(x-\mu)^T\big],\quad \mu=\mathbb{E}[x].
\]
$C$ is symmetric, PSD; diagonal entries are variances, off-diagonals are covariances.

\defn{Biased vs unbiased estimator:}
An estimator $\hat\theta$ of parameter $\theta$ is
\[
\text{unbiased if }\ \mathbb{E}[\hat\theta]=\theta,\quad \text{biased otherwise}.
\]
Examples:
\begin{itemize}[leftmargin=*]
\item Sample mean: $\hat\mu=\frac1N\sum_{i=1}^N X_i$ is unbiased for $\mu$.
\item Sample variance: $\hat\sigma^2=\frac1{N-1}\sum_i (X_i-\hat\mu)^2$ is unbiased; with $\frac1N$ it is biased.
\end{itemize}

\defn{Central Limit Theorem (CLT):}
If $X_i$ are IID with mean $\mu$ and variance $\sigma^2<\infty$, then for large $N$,
\[
\bar X_N=\frac1N\sum_{i=1}^N X_i \approx \mathcal{N}\Big(\mu,\frac{\sigma^2}{N}\Big).
\]
Only the \emph{sample mean} becomes approximately normal; the histogram of raw samples approaches the true underlying distribution, not necessarily Gaussian.

\ex{Example (law of total prob.):}
Events $A_i$ partition and event $B$:
\begin{equation*}
P(B)=\sum_i P(B|A_i)P(A_i).
\end{equation*}

%%%%%%%%%%%%%%%%%%%%%%%%%%%%%%%%%%%%%%%%%%%%%%%%%%%%
\subsection*{Notable distributions}
%%%%%%%%%%%%%%%%%%%%%%%%%%%%%%%%%%%%%%%%%%%%%%%%%%%%

\textbf{Bernoulli} (discrete):
\[
P(X=1)=p,\quad P(X=0)=1-p,\quad X\in\{0,1\}.
\]
Mean $\mathbb{E}X=p$, variance $\mathrm{Var}(X)=p(1-p)$.

\textbf{Binomial} (discrete):
\[
P(X=k)=\binom{n}{k}p^k(1-p)^{n-k},\quad k=0,1,\dots,n.
\]
Mean $np$, variance $np(1-p)$.

\textbf{Geometric} (discrete, ``\# of failures before first success'' version):
\[
P(X=k)=p(1-p)^k,\quad k=0,1,2,\dots
\]
Mean $\frac{1-p}{p}$, variance $\frac{1-p}{p^2}$.

\textbf{Poisson} (discrete):
\[
P(K=k)=\frac{\lambda^k e^{-\lambda}}{k!},\quad k=0,1,2,\dots
\]
Mean $\lambda$, variance $\lambda$.

\textbf{Uniform} (continuous on $[a,b]$):
\[
f_X(x)=\begin{cases}
\frac{1}{b-a}, & a\le x\le b,\\
0, & \text{otherwise},
\end{cases}
\]
Mean $\frac{a+b}{2}$, variance $\frac{(b-a)^2}{12}$.

\textbf{Normal / Gaussian} (continuous):
\[
f_X(x)=\frac{1}{\sqrt{2\pi\sigma^2}}\exp\!\left(-\frac{(x-\mu)^2}{2\sigma^2}\right),\quad x\in\mathbb{R}.
\]
Mean $\mu$, variance $\sigma^2$.

\textbf{Multivariate Normal} (continuous):
For $x\in\mathbb{R}^k$,
\[
p_X(x)=\frac{1}{(2\pi)^{k/2}\sqrt{\det C}}
\exp\!\left(-\tfrac12 (x-\mu)^T C^{-1}(x-\mu)\right),
\]
mean vector $\mu$, covariance $C$.

\textbf{Exponential} (continuous, rate $\lambda>0$):
\[
f_X(x)=\lambda e^{-\lambda x},\quad x\ge 0.
\]
Mean $1/\lambda$, variance $1/\lambda^2$.

\textbf{Gamma} (continuous, shape $\alpha>0$, rate $\beta>0$):
\[
f_X(x)=\frac{\beta^\alpha}{\Gamma(\alpha)}x^{\alpha-1}e^{-\beta x},\quad x\ge 0.
\]
Mean $\alpha/\beta$, variance $\alpha/\beta^2$.

\noindent\hrule
\vspace{2pt}
\pagebreak


%%%%%%%%%%%%%%%%%%%%%%%%%%%%%%%%%%%%%%%%%%%%%%%%%%%%
\section*{5. Models vs data \& design}
\noindent\hrule
\vspace{1pt}
%%%%%%%%%%%%%%%%%%%%%%%%%%%%%%%%%%%%%%%%%%%%%%%%%%%%

\subsection*{Linear models and estimation}

\defn{Design matrix $X$:} rows are input vectors $\vect{x}_i^T$, columns correspond to parameters.

Linear model (scalar output):
\begin{equation*}
y_i = \vect{x}_i^T\bm{\theta} + \varepsilon_i,\quad
\varepsilon_i\sim \mathcal{N}(0,\sigma^2),
\end{equation*}
\begin{equation*}
\vect{y}=X\bm{\theta}+\bm{\varepsilon}.
\end{equation*}

\defn{Homoscedastic vs heteroscedastic errors:}
\begin{itemize}[leftmargin=*]
\item Homoscedastic: $\mathrm{Var}(\varepsilon_i)=\sigma^2$ for all $i$ (same variance).
\item Heteroscedastic: $\mathrm{Var}(\varepsilon_i)$ depends on $i$ (or on $\vect{x}_i$).
\end{itemize}

\textbf{OLS (ordinary least squares):} homoscedastic, uncorrelated errors.
\begin{equation*}
\hat{\bm{\theta}}_{\mathrm{OLS}}
= \arg\min_{\theta}\sum_i (y_i - \vect{x}_i^T\theta)^2
= (X^TX)^{-1}X^T\vect{y}.
\end{equation*}

\textbf{WLS (weighted least squares):} known covariance $V$ of $\varepsilon$.
\begin{equation*}
\hat{\bm{\theta}}_{\mathrm{WLS}}
= \arg\min_{\theta}(\vect{y}-\vect{y}^{\text{model}}(\theta))^T
V^{-1}(\vect{y}-\vect{y}^{\text{model}}(\theta)).
\end{equation*}

If errors are Gaussian, OLS/WLS coincide with MLE:

\defn{Likelihood and MLE:}
\begin{equation*}
L(\theta)=p(D\mid\theta),\quad
\hat\theta_{\mathrm{MLE}}=\arg\max_{\theta} L(\theta)
=\arg\max_{\theta}\log L(\theta).
\end{equation*}
For independent Gaussian errors with covariance $V$,
\begin{equation*}
\log L(\theta)
= -\tfrac12(\vect{y}-\vect{y}^{\text{model}}(\theta))^T
V^{-1}(\vect{y}-\vect{y}^{\text{model}}(\theta)) + \text{const.}
\end{equation*}
$\Rightarrow$ maximizing likelihood $\equiv$ minimizing (weighted) SSE.

\ex{Example (OLS fit):}
Data $(x_i,y_i)$, linear model $y=ax+b$.
\begin{equation*}
X=
\begin{bmatrix}
1 & x_1\\
\vdots & \vdots\\
1 & x_N
\end{bmatrix},\quad
\hat{\bm{\theta}}=
\begin{bmatrix}\hat{b}\\ \hat{a}\end{bmatrix}
=(X^TX)^{-1}X^T\vect{y}.
\end{equation*}

\defn{Maximum Likelihood Estimation (MLE):}
Given data $x_1,\dots,x_N$ assumed IID from a model with parameter $\theta$, the \emph{likelihood} is
\[
L(\theta)=\prod_{i=1}^N p(x_i\mid\theta),
\]
viewed as a function of $\theta$. The MLE chooses the parameter value that makes the observed data most probable:
\[
\hat\theta_{\mathrm{MLE}}
=\operatorname*{arg\,max}_{\theta} L(\theta)
=\operatorname*{arg\,max}_{\theta} \prod_{i=1}^N p(x_i\mid\theta).
\]
Equivalently, because $\log$ is monotone,
\[
\hat\theta_{\mathrm{MLE}}
=\operatorname*{arg\,max}_{\theta} \sum_{i=1}^N \log p(x_i\mid\theta),
\]
which is usually easier to optimize numerically. Intuition: pick the parameter that \emph{best explains the data} under the assumed probabilistic model.


\subsection*{Bayesian vs frequentist, MAP, and Bayes' theorem}

\defn{Frequentist view:}
$\theta$ is fixed but unknown; uncertainty is in data. Use point estimates (OLS/WLS/MLE) and confidence or $\chi^2$ regions.

\defn{Bayesian view:}
$\theta$ is random with prior $p(\theta)$; data update beliefs via Bayes' theorem to posterior $p(\theta\mid D)$; use MAP and credible regions.

\defn{Bayes' theorem (parameters):}
\begin{equation*}
p(\theta\mid D)
=\frac{p(D\mid\theta)p(\theta)}{p(D)},\quad
p(D)=\int p(D\mid\theta)p(\theta)\,d\theta.
\end{equation*}
Posterior $\propto$ likelihood $\times$ prior.

\defn{Prior/posterior/MAP:}
\begin{itemize}[leftmargin=*]
\item Prior $p(\theta)$: belief about $\theta$ before data.
\item Posterior $p(\theta\mid D)\propto p(D\mid\theta)p(\theta)$.
\item MAP estimator:
\[
\hat\theta_{\mathrm{MAP}}
=\arg\max_{\theta} p(\theta\mid D)
=\arg\max_{\theta}[\log L(\theta)+\log p(\theta)].
\]
\end{itemize}
Connections:
\begin{itemize}[leftmargin=*]
\item OLS/WLS/MLE: minimize SSE or weighted SSE; frequentist.
\item MAP: regularized SSE (log prior acts as penalty); Bayesian.
\end{itemize}

\defn{Conjugate prior:}
A prior family $p(\theta)$ is conjugate to a likelihood $p(D\mid\theta)$ if the posterior $p(\theta\mid D)$ is in the same family. Example: $\lambda\sim\mathrm{Gamma}(\alpha,\beta)$ and $k\mid\lambda\sim\mathrm{Poisson}(\lambda)$
$\Rightarrow$ $\lambda\mid k\sim\mathrm{Gamma}(\alpha+k,\beta+1)$.

\subsection*{Fisher information, covariance, and design}

\defn{Fisher information:}
\begin{equation*}
\mathcal{I}(\theta)
= -\mathbb{E}\big[\nabla^2_{\theta}\log L(\theta)\big].
\end{equation*}

For linear, homoscedastic model $\vect{y}=X\theta+\varepsilon$, $\varepsilon\sim\mathcal{N}(0,\sigma^2 I)$:
\begin{equation*}
\widehat{\mathrm{Cov}}(\hat{\bm{\theta}}) \approx \sigma^2 (X^TX)^{-1},\quad
\mathcal{I}(\theta) = \frac{1}{\sigma^2}X^TX.
\end{equation*}

For general nonlinear models with sensitivity Jacobian $J(\theta)$ and error covariance $V$:
\begin{equation*}
\mathcal{I}(\theta)\approx J(\theta)^T V^{-1} J(\theta).
\end{equation*}

\defn{Experimental design:}
Choose inputs (rows of $X$ or evaluations of $J$) to reduce parameter uncertainty by optimizing a scalar function of the information matrix.

Let $M$ denote an information matrix (e.g.\ $M=X^TX$ or $M=J^TJ$ for homoscedastic errors). Common criteria:
\begin{itemize}[leftmargin=*]
\item A-optimal: minimize $\mathrm{tr}(M^{-1})$ (average parameter variance).
\item D-optimal: maximize $\det(M)$ (minimize volume of confidence ellipsoid).
\item E-optimal: minimize largest eigenvalue of $M^{-1}$ (worst-case variance).
\end{itemize}

\subsection*{$\chi^2$, region of indifference, and credible regions}

\defn{$\chi^2$ misfit (Gaussian errors):}
If measurement covariance is $V$,
\begin{equation*}
\chi^2(\theta)
= (\vect{y}-\vect{y}^{\text{model}}(\theta))^T
V^{-1}(\vect{y}-\vect{y}^{\text{model}}(\theta)).
\end{equation*}
Under correct Gaussian noise, $\chi^2(\theta_{\text{true}})$ approximately follows a $\chi^2$ distribution with
\[
\nu = (\text{\# data}) - (\text{\# fitted parameters}).
\]

\defn{Region of indifference (frequentist):}
For significance level $\alpha$,
\begin{equation*}
\mathcal{R}_\alpha
= \{\theta:\ \chi^2(\theta)\le \chi^2_{\max}\},
\end{equation*}
where $\chi^2_{\max}$ is a chosen quantile of the $\chi^2_\nu$ distribution (e.g.\ $1-\alpha$). Parameters in $\mathcal{R}_\alpha$ give fits consistent with the data at level $\alpha$.

\defn{Credible interval/region (Bayesian):}
A $(1-\alpha)$ credible set $C$ satisfies
\begin{equation*}
P(\theta\in C\mid D)=1-\alpha.
\end{equation*}
If $p(\theta\mid D)\approx \mathcal{N}(\hat\theta, V)$, an ellipsoidal credible region is
\begin{equation*}
(\theta-\hat\theta)^T V^{-1}(\theta-\hat\theta)\le c_\alpha,
\end{equation*}
with $c_\alpha$ chosen to contain posterior mass $1-\alpha$. This quadratic form is analogous to a $\chi^2$ threshold (but interpreted Bayesianly).

\noindent\hrule
\vspace{2pt}

%\pagebreak


%%%%%%%%%%%%%%%%%%%%%%%%%%%%%%%%%%%%%%%%%%%%%%%%%%%%
\section*{6. Monte Carlo \& MCMC}
\noindent\hrule
\vspace{1pt}
%%%%%%%%%%%%%%%%%%%%%%%%%%%%%%%%%%%%%%%%%%%%%%%%%%%%

\subsection*{Monte Carlo integration}
\defn{Monte Carlo integration (simple sampling):}
Approximate an integral by averaging random samples.
\begin{equation*}
I=\int_{\Omega} f(\vect{x})\,d\vect{x}
= \mathrm{Vol}(\Omega)\,\mathbb{E}[f(X)],\ X\sim \mathrm{Unif}(\Omega),
\end{equation*}
\begin{equation*}
I \approx \frac{\mathrm{Vol}(\Omega)}{N}\sum_{i=1}^N f(\vect{x}_i),
\quad \vect{x}_i \sim \mathrm{Unif}(\Omega).
\end{equation*}
Error scales as $O(N^{-1/2})$, independent of dimension.

\defn{Importance sampling:}
Rewrite
\begin{equation*}
I = \int f(x)\,dx
= \int f(x)\,\frac{p(x)}{p(x)}\,dx
= \mathbb{E}_{p}\!\left[\frac{f(X)}{p(X)}\right],
\end{equation*}
where $p(x)>0$ wherever $f(x)\neq 0$.
Estimate:
\begin{equation*}
I \approx \frac{1}{N}\sum_{i=1}^N \frac{f(x_i)}{p(x_i)},\quad x_i\sim p.
\end{equation*}
Choosing $p$ with similar shape to $f$ reduces variance.

\subsection*{Normalized vs unnormalized densities}

\defn{Normalized vs unnormalized:}
\begin{itemize}[leftmargin=*]
\item A normalized pdf $\pi(x)$ satisfies $\int \pi(x)\,dx=1$.
\item Often only known up to a constant: $\tilde\pi(x)\propto \pi(x)$.
\end{itemize}
In many computations (e.g.\ Metropolis–Hastings), the normalization constant cancels:
ratios $\pi(x')/\pi(x)$ can be computed using $\tilde\pi$.

Examples:
\begin{itemize}[leftmargin=*]
\item Posterior: $\pi(\theta\mid D)\propto p(D\mid\theta)p(\theta)$, evidence $p(D)$ unknown.
\item Target for MCMC: only need $\tilde\pi$ for acceptance probability.
\end{itemize}

\subsection*{Markov chains and MCMC}

\defn{Markov chain:}
Sequence $X_0,X_1,\dots$ with
\begin{equation*}
P(X_{k+1}\in A\mid X_k=x_k,\dots,X_0=x_0) 
= P(X_{k+1}\in A\mid X_k=x_k).
\end{equation*}

\defn{Target / stationary distribution:}
A distribution $\pi$ is stationary for the chain if
\[
X_0\sim \pi \ \Rightarrow\ X_k\sim \pi\ \forall k.
\]
A sufficient condition is \emph{detailed balance}:
\[
\pi(x)T(x\to x') = \pi(x')T(x'\to x)\quad\forall x,x',
\]
where $T(x\to x')$ is the one-step transition density/mass.

\defn{MCMC:}
Construct a Markov chain with stationary distribution $\pi(x)$ (often known only up to a constant), then approximate expectations
\[
\mathbb{E}_\pi[g(X)] = \int g(x)\pi(x)\,dx
\approx \frac{1}{N}\sum_{i=B+1}^{B+N} g(x_i),
\]
where $\{x_i\}$ is the chain, and $B$ is burn-in.

\defn{Markov property (first-order chain):}
A discrete-time stochastic process $\{X_0,X_1,\dots,X_n\}$ is a (first-order) Markov chain if for all $k$:
\[
P(X_{k+1}=x_{k+1}\mid X_k=x_k, X_{k-1}=x_{k-1},\dots,X_0=x_0) =...
\]
\[
...= P(X_{k+1}=x_{k+1}\mid X_k=x_k)
\]
Intuition: the future depends on the present, not on the full past (“one-step memory”). 

\defn{Markov chain joint factorization:}
For a Markov chain $X_0,\dots,X_n$,
\[
P(X_0=x_0,\dots,X_n=x_n)=...
\]
\[
...= P(X_0=x_0)\,\prod_{k=1}^n P(X_k=x_k\mid X_{k-1}=x_{k-1}).
\]
If the chain is time-homogeneous with transition matrix $P$, then
$P(X_k=j\mid X_{k-1}=i)=P_{ij}$ and the joint is fully determined by the initial distribution and $P$. 

\defn{Bayesian network (directed Markov factorization):}
Let $X=(X_1,\dots,X_n)$ be random variables with a directed acyclic graph (DAG) where $\mathrm{Pa}(i)$ are the parents of node $i$. If the distribution is Markov w.r.t.\ this DAG, its joint pdf/pmf factorizes as
\[
p(x_1,\dots,x_n)=\prod_{i=1}^n p\bigl(x_i \mid x_{\mathrm{Pa}(i)}\bigr).
\]
Intuition: each node only “looks at” its parents; all other conditional independencies follow from the graph.



\subsection*{Metropolis–Hastings (MH) algorithm}

Given current state $x_k$ and target $\pi(x)$ (we can use unnormalized $\tilde\pi$):

\begin{enumerate}[leftmargin=*]
\item Propose $x'\sim q(\cdot\mid x_k)$ (proposal kernel).
\item Compute acceptance probability
\[
\alpha(x',x_k)=\min\left(1,
\frac{\tilde\pi(x')\,q(x_k\mid x')}{\tilde\pi(x_k)\,q(x'\mid x_k)}
\right).
\]
\item Draw $u\sim \mathrm{Unif}(0,1)$; if $u\le \alpha$, set $x_{k+1}=x'$, else $x_{k+1}=x_k$.
\end{enumerate}

If $q$ is symmetric, $q(x'\mid x)=q(x\mid x')$, then
\[
\alpha(x',x)=\min\left(1,\frac{\tilde\pi(x')}{\tilde\pi(x)}\right).
\]

\defn{Using MCMC for expectations:}
If the chain is ergodic (can reach all relevant states and has unique stationary distribution),
\[
\frac{1}{N}\sum_{i=B+1}^{B+N} g(x_i) \to \mathbb{E}_\pi[g(X)]
\]
as $N\to\infty$.

\subsection*{Practical points}

\begin{itemize}[leftmargin=*]
\item \textbf{Burn-in:} discard initial samples before chain reaches stationarity.
\item \textbf{Thinning:} (optional) keep every $k$-th sample to reduce autocorrelation; not strictly necessary if $N$ is large.
\item \textbf{Ergodicity/irreducibility:} proposal must allow reaching all regions where $\pi(x)>0$; avoid one-sided proposals that can never go backward.
\item \textbf{Unnormalized targets:} for posterior $p(\theta\mid D)\propto p(D\mid\theta)p(\theta)$, use $\tilde\pi(\theta)=p(D\mid\theta)p(\theta)$ directly in MH; no need to compute evidence $p(D)$.
\end{itemize}




\begin{lstlisting}
function samples = metropolis_hastings(logpdf, x0, nsamples, proposal_std)
% METROPOLIS_HASTINGS  Basic Metropolis-Hastings sampler
%
% Inputs:
%   logpdf        - function handle, logpdf(x) returns log p(x) up to a constant
%   x0            - initial state (column vector or scalar)
%   nsamples      - number of samples to draw
%   proposal_std  - scalar std dev for Gaussian random-walk proposal
%
% Output:
%   samples       - matrix of samples, size = [length(x0), nsamples]
    % Preallocate storage for samples
    dim     = numel(x0);
    samples = zeros(dim, nsamples);

    % Current state and its log-density
    x_curr     = x0(:);               % ensure column vector
    logp_curr  = logpdf(x_curr);

    for k = 1:nsamples

        % --- PROPOSE A NEW STATE (Gaussian random walk) ---
        % For multivariate x, use independent N(0, proposal_std^2) in each dim
        x_prop = x_curr + proposal_std * randn(dim, 1);

        % --- COMPUTE ACCEPTANCE RATIO ---
        % For symmetric proposals q(x'|x) = q(x|x'), Hastings term cancels
        logp_prop = logpdf(x_prop);
        % log acceptance ratio: log alpha = log p(x_prop) - log p(x_curr)
        log_alpha = logp_prop - logp_curr;
        % Draw u ~ Uniform(0,1) and accept if log u < log alpha
        if log(rand) < log_alpha
            % Accept proposal
            x_curr    = x_prop;
            logp_curr = logp_prop;
        end

        % Store current state (whether moved or not)
        samples(:, k) = x_curr;
    end
end
\end{lstlisting}



\noindent\hrule
\vspace{2pt}

% \pagebreak


%%%%%%%%%%%%%%%%%%%%%%%%%%%%%%%%%%%%%%%%%%%%%%%%%%%%
\section*{7. ODE IVPs}
\noindent\hrule
\vspace{1pt}
%%%%%%%%%%%%%%%%%%%%%%%%%%%%%%%%%%%%%%%%%%%%%%%%%%%%

\defn{ODE IVP:} find $\mathbf{x}(t)$ such that
\[
\dot{\mathbf{x}} = f(t,\mathbf{x}), \quad \mathbf{x}(t_0)=\mathbf{x}_0.
\]

\subsection*{Time-stepping methods}

\textbf{Explicit Euler (Forward Euler):}
\[
\mathbf{x}_{n+1} = \mathbf{x}_n + \Delta t\, f(t_n,\mathbf{x}_n).
\]

\textbf{Implicit Euler (Backward Euler):}
\[
\mathbf{x}_{n+1} = \mathbf{x}_n + \Delta t\, f(t_{n+1},\mathbf{x}_{n+1}),
\]
requires solving a nonlinear system each step (e.g.\ Newton). Good stability for stiff problems.

\textbf{Midpoint / RK2 (explicit):}
\[
\mathbf{k}_1 = f(t_n,\mathbf{x}_n),\quad
\mathbf{k}_2 = f\Bigl(t_n+\tfrac{\Delta t}{2},\,\mathbf{x}_n+\tfrac{\Delta t}{2}\mathbf{k}_1\Bigr),
\]
\[
\mathbf{x}_{n+1} = \mathbf{x}_n + \Delta t\,\mathbf{k}_2.
\]
Higher order than Euler for similar cost. 

\textbf{Classical RK4 (explicit):}
\[
\begin{aligned}
\mathbf{k}_1 &= f(t_n,\mathbf{x}_n),\\
\mathbf{k}_2 &= f\Bigl(t_n+\tfrac{\Delta t}{2},\,\mathbf{x}_n+\tfrac{\Delta t}{2}\mathbf{k}_1\Bigr),\\
\mathbf{k}_3 &= f\Bigl(t_n+\tfrac{\Delta t}{2},\,\mathbf{x}_n+\tfrac{\Delta t}{2}\mathbf{k}_2\Bigr),\\
\mathbf{k}_4 &= f\bigl(t_n+\Delta t,\,\mathbf{x}_n+\Delta t\,\mathbf{k}_3\bigr),\\
\mathbf{x}_{n+1} &= \mathbf{x}_n + \frac{\Delta t}{6}\bigl(\mathbf{k}_1+2\mathbf{k}_2+2\mathbf{k}_3+\mathbf{k}_4\bigr).
\end{aligned}
\]

\subsection*{Accuracy and local truncation error}

\defn{Local truncation error (LTE):} error made in one step assuming $\mathbf{x}_n$ is exact.
\begin{itemize}[leftmargin=*]
\item Explicit Euler: LTE $=O(\Delta t^2)$, global error $=O(\Delta t)$ (1st order).
\item Midpoint / RK2: LTE $=O(\Delta t^3)$, global error $=O(\Delta t^2)$ (2nd order). 
\item RK4: LTE $=O(\Delta t^5)$, global error $=O(\Delta t^4)$ (4th order). 
\end{itemize}

\subsection*{Stability and stiffness}

Test problem: $\dot{x}=\lambda x$, exact solution $x(t)=e^{\lambda t}x_0$.

\defn{Stability region:} set of $z=\lambda\Delta t$ such that the numerical solution does not grow for $\operatorname{Re}(\lambda)<0$. For a one-step method $x_{n+1}=R(z)x_n$, stability requires $|R(z)|\le 1$. 

Examples:
\begin{itemize}[leftmargin=*]
\item Explicit Euler: $R(z)=1+z$, stability if $|1+z|\le 1$. For real $\lambda<0$, this gives $-2\le \lambda\Delta t \le 0$.
\item Implicit Euler: $R(z)=1/(1-z)$; $|R(z)|<1$ for all $\operatorname{Re}(z)<0$ (A-stable). Good for stiff problems. 
\end{itemize}

\defn{Stiffness:} the system has widely separated time scales (eigenvalues with very different negative real parts). Explicit methods require very small $\Delta t$ for stability; implicit methods allow much larger steps without instability (though accuracy still limits $\Delta t$). 

\ex{Explicit Euler on decay:}
\[
\dot{x} = -k x,\quad x(0)=x_0.
\]
Euler:
\[
x_{n+1} = x_n - k\Delta t\,x_n = (1-k\Delta t)x_n.
\]
Stable if $|1-k\Delta t|\le 1 \Rightarrow 0\le k\Delta t\le 2$.

\noindent\hrule
\vspace{2pt}

%%%%%%%%%%%%%%%%%%%%%%%%%%%%%%%%%%%%%%%%%%%%%%%%%%%%
\section*{8. ODE BVPs}
\noindent\hrule
\vspace{1pt}
%%%%%%%%%%%%%%%%%%%%%%%%%%%%%%%%%%%%%%%%%%%%%%%%%%%%

\defn{BVP (boundary value problem):} differential equation with conditions at multiple points (typically interval endpoints). Example:
\[
y''(x) = F(x,y,y'),\quad x\in[a,b],\quad
\]
\[
\alpha_1(y(a),y'(a)) = 0,\ \alpha_2(y(b),y'(b)) = 0.
\]

\subsection*{Shooting method (single shooting)}

Idea: turn BVP into an IVP with unknown initial condition(s).
\begin{itemize}[leftmargin=*]
\item Introduce unknown initial slope/parameters $\mathbf{c}$ (e.g.\ $y'(a)=c$).
\item Solve IVP for given $\mathbf{c}$ to get $y(x;\mathbf{c})$.
\item Evaluate boundary residual at $x=b$: $\mathbf{h}(\mathbf{c})$.
\item Use a root finder (Newton or secant) to solve $\mathbf{h}(\mathbf{c})=0$.
\end{itemize}
Works well when IVP is well-behaved; can struggle for highly sensitive/unstable IVPs. 

\ex{Example (linear BVP):}
\[
y''=0,\quad y(0)=0,\ y(1)=1.
\]
Let $y'(0)=c$. Solve IVP:
$y(x)=cx$. Enforce $y(1)=1\Rightarrow c=1$, so $y(x)=x$.

\subsection*{Finite difference method (FDM) for BVPs}

Example:
\[
A\,y''(x) = f(x),\quad x\in[a,b],
\]
with Dirichlet BCs $y(a)=\alpha$, $y(b)=\beta$.

Mesh: $x_i=a+i\Delta x$, $i=0,\dots,M+1$ with $\Delta x=(b-a)/(M+1)$.
Use interior points $i=1,\dots,M$.

Centered second derivative:
\[
y''(x_i) \approx \frac{y_{i+1}-2y_i+y_{i-1}}{\Delta x^2}.
\]
Leads to a tridiagonal linear system $A_h \mathbf{y}=\mathbf{b}$ for unknown interior values. 

BC handling:
\begin{itemize}[leftmargin=*]
\item Dirichlet: set $y_0=\alpha$, $y_{M+1}=\beta$ directly.
\item Neumann: approximate derivative with one-sided or ghost-point finite differences.
\end{itemize}

\textbf{Remarks:}
\begin{itemize}[leftmargin=*]
\item Shooting leverages IVP solvers (adaptive time stepping) but can suffer from sensitivity.
\item Finite difference BVP is more robust and leads to linear/nonlinear algebraic systems. 
\end{itemize}

\noindent\hrule
\vspace{2pt}

\columnbreak

%%%%%%%%%%%%%%%%%%%%%%%%%%%%%%%%%%%%%%%%%%%%%%%%%%%%
\section*{9. PDEs}
\noindent\hrule
\vspace{1pt}
%%%%%%%%%%%%%%%%%%%%%%%%%%%%%%%%%%%%%%%%%%%%%%%%%%%%

\subsection*{Classification}

General 2D second-order linear PDE:
\[
A u_{xx} + B u_{xy} + C u_{yy} + D u_x + E u_y + Gu = H.
\]
Classification via $B^2 - 4AC$:
\begin{itemize}[leftmargin=*]
\item Elliptic: $B^2 - 4AC < 0$ (e.g.\ Laplace, Poisson).
\item Parabolic: $B^2 - 4AC = 0$ (e.g.\ heat equation).
\item Hyperbolic: $B^2 - 4AC > 0$ (e.g.\ wave, advection). 
\end{itemize}

\subsection*{Finite difference for elliptic PDEs}

Example (Laplace):
\[
u_{xx}+u_{yy}=0 \quad \text{in a rectangle}.
\]
Uniform grid with spacing $h$ in both directions; interior node $(i,j)$:
\[
u_{xx}(x_i,y_j) \approx \frac{u_{i+1,j}-2u_{i,j}+u_{i-1,j}}{h^2},\quad
\]
\[
u_{yy}(x_i,y_j) \approx \frac{u_{i,j+1}-2u_{i,j}+u_{i,j-1}}{h^2}.
\]
5-point stencil:
\[
u_{i+1,j}+u_{i-1,j}+u_{i,j+1}+u_{i,j-1}-4u_{i,j} = 0.
\]
Leads to a sparse linear system $A\mathbf{u}=\mathbf{b}$. 

\subsection*{Method of lines (MoL) for time-dependent PDEs}

\defn{Method of lines:} discretize space (e.g.\ by finite differences) to obtain a large ODE system in time; then apply ODE IVP solvers. 

Example: 1D diffusion
\[
u_t = D u_{xx},\quad x\in[0,L],
\]
space discretization:
\[
\frac{d}{dt} u_j(t) = D \frac{u_{j+1}(t)-2u_j(t)+u_{j-1}(t)}{\Delta x^2},
\]
which is a stiff ODE system that can be integrated with implicit methods or \texttt{ode15s}.

\subsection*{Advection, CFL, and upwinding}

Example advection:
\[
u_t + a u_x = 0,\quad a>0.
\]

\defn{CFL number:} 
\[
\mathrm{CFL} = \frac{a\Delta t}{\Delta x}.
\]
Roughly, Courant–Friedrichs–Lewy condition requires the numerical domain of dependence to contain the PDE’s domain of dependence; gives step restrictions for explicit schemes. 

\textbf{Upwind scheme (first-order):}
For $a>0$,
\[
u_j^{n+1} = u_j^n - \mathrm{CFL}\,(u_j^n - u_{j-1}^n).
\]
This is stable for $0 \le \mathrm{CFL} \le 1$. 

\textbf{Central difference in space + forward Euler in time:}
\[
u_j^{n+1} = u_j^n - \frac{\mathrm{CFL}}{2}\,(u_{j+1}^n - u_{j-1}^n)
\]
is typically unstable for pure advection, motivating upwinding and/or more advanced schemes.

\ex{Example (CFL check):}
$a=2$, $\Delta x=0.1$, want $\mathrm{CFL}=1$:
\[
\mathrm{CFL} = \frac{a\Delta t}{\Delta x} = 1 \Rightarrow \Delta t = 0.05.
\]

\noindent\hrule
\vspace{2pt}


%%%%%%%%%%%%%%%%%%%%%%%%%%%%%%%%%%%%%%%%%%%%%%%%%%%%
\section*{10. Differential-Algebraic Equations (DAEs)}
\noindent\hrule
\vspace{1pt}
%%%%%%%%%%%%%%%%%%%%%%%%%%%%%%%%%%%%%%%%%%%%%%%%%%%%

\defn{DAE:} A system that mixes differential and algebraic equations.

\textbf{Semi-explicit form:}
\begin{equation*}
\dot{\mathbf{x}} = f(t,\mathbf{x},\mathbf{y}), \quad
0 = g(t,\mathbf{x},\mathbf{y}),
\end{equation*}
where $\mathbf{x}$ are differential states (time derivatives appear) and
$\mathbf{y}$ are algebraic states (no time derivatives).

\textbf{Implicit form:}
\begin{equation*}
F\bigl(t,\mathbf{z},\dot{\mathbf{z}}\bigr) = 0,
\end{equation*}
where $\mathbf{z}$ collects all states. If $\partial F / \partial \dot{\mathbf{z}}$ is singular,
the system is a genuine DAE (not just an implicit ODE).

\textbf{Mass-matrix form:}
\begin{equation*}
M(t,\mathbf{y})\,\dot{\mathbf{y}} = F(t,\mathbf{y}),
\end{equation*}
with (possibly singular) mass matrix $M$. If $M$ is nonsingular, this is an ODE;
if $M$ is singular, it is a DAE suitable for mass-matrix solvers (e.g.\ \texttt{ode15s}).

\defn{Differential index:}
The smallest number of times the algebraic equations must be differentiated
(with respect to time) to obtain an explicit ODE for all derivatives
$\dot{\mathbf{x}},\dot{\mathbf{y}}$.

\textbf{Index-1 test (semi-explicit):}
Given
\[
\dot{\mathbf{x}} = f(t,\mathbf{x},\mathbf{y}), \quad
0 = g(t,\mathbf{x},\mathbf{y}),
\]
differentiate $g=0$ once:
\[
\frac{d}{dt}g(t,\mathbf{x},\mathbf{y})
= g_t + g_{\mathbf{x}} \dot{\mathbf{x}} + g_{\mathbf{y}} \dot{\mathbf{y}} = 0.
\]
If $g_{\mathbf{y}}$ is nonsingular, solve for $\dot{\mathbf{y}}$:
\[
\dot{\mathbf{y}} = -\,g_{\mathbf{y}}^{-1}\bigl(g_t + g_{\mathbf{x}} \dot{\mathbf{x}}\bigr),
\]
so the system is index-1. If further differentiation is needed to isolate derivatives,
the index is $>1$.

\defn{Implicit (hidden) constraints:}
Higher-index DAEs have additional constraints obtained by differentiating
$g(t,\mathbf{x},\mathbf{y})=0$ (e.g.\ $\dot g=0$, $\ddot g=0$).
Solutions must satisfy both the explicit algebraic constraints $g=0$ and these
derivative constraints. Index-1 DAEs typically do not have nontrivial hidden constraints.

\defn{Consistent initialization:}
An initial triple $(\mathbf{x}_0,\dot{\mathbf{x}}_0,\mathbf{y}_0)$ that satisfies
\begin{itemize}[leftmargin=*]
\item the DAE at $t_0$:
\[
\dot{\mathbf{x}}_0 = f(t_0,\mathbf{x}_0,\mathbf{y}_0), \quad
0 = g(t_0,\mathbf{x}_0,\mathbf{y}_0),
\]
\item and, for higher-index systems, any implied derivative constraints
obtained by differentiating $g=0$ enough times (e.g.\ $\dot g(t_0,\mathbf{x}_0,\mathbf{y}_0,\dot{\mathbf{x}}_0,\dot{\mathbf{y}}_0)=0$).
\end{itemize}
Not every choice of $(\mathbf{x}_0,\mathbf{y}_0)$ is allowed; algebraic and hidden
constraints restrict admissible initial conditions.

\ex{Example (semi-explicit index-1):}
\begin{equation*}
\dot{x} = y, \quad 0 = x + y.
\end{equation*}
Algebraic constraint: $x + y = 0 \Rightarrow y = -x$.
Substitute into the differential equation:
\[
\dot{x} = -x,
\]
which is an ODE. The system is index-1. Consistent initialization requires $y_0 = -x_0$.

\defn{Variables, equations, and DoF (semi-explicit index-1):}
Consider
\[
\dot{\mathbf{x}} = f(t,\mathbf{x},\mathbf{y}), \quad 0 = g(t,\mathbf{x},\mathbf{y}),
\]
with differential states $\mathbf{x}\in\mathbb{R}^{n_d}$, algebraic states $\mathbf{y}\in\mathbb{R}^{n_a}$, and $g_{\mathbf{y}}$ full rank so that $g$ provides $n_a$ independent algebraic equations. At $t_0$:
\begin{itemize}[leftmargin=*]
\item Unknowns in the initialization: $(\mathbf{x}_0,\mathbf{y}_0)\in\mathbb{R}^{n_d+n_a}$.
\item Algebraic constraints: $g(t_0,\mathbf{x}_0,\mathbf{y}_0)=0$ impose $n_a$ independent conditions.
\end{itemize}
Thus the number of degrees of freedom in specifying a consistent initialization is
\[
\text{DoF} = (n_d + n_a) - n_a = n_d.
\]
A convenient viewpoint for regular index-1 systems is:
\begin{itemize}[leftmargin=*]
\item choose the $n_d$ components of $\mathbf{x}_0$ freely,
\item then solve $g(t_0,\mathbf{x}_0,\mathbf{y}_0)=0$ for the $n_a$ components of $\mathbf{y}_0$ to obtain a consistent $(\mathbf{x}_0,\mathbf{y}_0)$.
\end{itemize}
For higher-index DAEs, additional hidden constraints from differentiated $g$ further reduce admissible initial values and may require specifying some derivatives at $t_0$ as well.

\defn{Index reduction (idea):}
Rewrite a higher-index DAE as an equivalent index-1 DAE by:
\begin{itemize}[leftmargin=*]
\item differentiating algebraic equations to expose hidden constraints,
\item eliminating redundant variables/constraints or introducing new variables
for derivatives,
\item reformulating the model in semi-explicit or mass-matrix index-1 form.
\end{itemize}
This facilitates numerical solution with standard DAE solvers.

\textbf{Connection to ODE-IVPs:}
\begin{itemize}[leftmargin=*]
\item Index-0 DAE $\equiv$ ODE-IVP: $\dot{\mathbf{x}} = f(t,\mathbf{x})$.
\item Index-1 DAEs can typically be solved directly with DAE-capable time-stepping
methods once consistent initial conditions are available.
\item Higher-index DAEs are usually reduced to index-1 before numerical solution.
\end{itemize}

\subsection*{Numerical solution in MATLAB (index-1 DAEs)}

\textbf{Mass-matrix solvers (\texttt{ode15s}, \texttt{ode23t}):}
Solve
\[
M(t,\mathbf{y})\,\dot{\mathbf{y}} = F(t,\mathbf{y}),
\]
with possibly singular $M$. You provide $M$, $F$, an initial $\mathbf{y}_0$
(satisfying algebraic constraints), and optionally $\dot{\mathbf{y}}_0$;
the solver can also attempt to determine a consistent $\dot{\mathbf{y}}_0$.

\textbf{Fully implicit solver (\texttt{ode15i}):}
Solve
\[
F(t,\mathbf{y},\dot{\mathbf{y}}) = 0
\]
for index-1 DAEs in implicit form. Typical workflow:
\begin{itemize}[leftmargin=*]
\item Define a function for $F(t,\mathbf{y},\dot{\mathbf{y}})$.
\item Use \texttt{decic} to compute consistent $(\mathbf{y}_0,\dot{\mathbf{y}}_0)$
from approximate guesses.
\item Call \texttt{ode15i} with $F$, $(\mathbf{y}_0,\dot{\mathbf{y}}_0)$, and the time grid.
\end{itemize}


%%%%%%%%%%%%%%%%%%%%%%%%%%%%%%%%%%%%%%%%%%%%%%%%%%%%
\noindent\hrule
\vspace{2pt}
%%%%%%%%%%%%%%%%%%%%%%%%%%%%%%%%%%%%%%%%%%%%%%%%%%%%


%%%%%%%%%%%%%%%%%%%%%%%%%%%%%%%%%%%%%%%%%%%%%%%%%%%%
\section*{11. Quick pattern-matching table}
\noindent\hrule
\vspace{1pt}
%%%%%%%%%%%%%%%%%%%%%%%%%%%%%%%%%%%%%%%%%%%%%%%%%%%%

\vspace{-4pt}
\begin{tabular}{|p{1.45in}|p{1.55in}|}
\hline
\textbf{Question type} & \textbf{Immediate move} \\
\hline
Solve \(Ax=b\) &
Check \(m,n\), \(\kappa(A)\); use backslash; think about uniqueness. \\
\hline
Nonlinear equations &
Form \(f(x)\), use Newton/tangent iteration; need Jacobian and stopping criteria. \\
\hline
Unconstrained min &
Write \(\nabla f\), \(H\); use Newton/gradient descent; check \(H\succ 0\). \\
\hline
Regression &
Build \(X\), \(y\); use OLS/WLS; compute \(\theta\), covariance, goodness-of-fit. \\
\hline
MLE / MAP &
Write likelihood/log-likelihood; differentiate, set to 0; add prior for MAP. \\
\hline
MC integral &
Express integral as expectation; choose proposal/target; define estimator. \\
\hline
ODE-IVP &
Identify stiffness; choose explicit vs implicit; argue stability and accuracy. \\
\hline
BVP &
BVP vs IVP; choose shooting or FD/FEM; formulate equations. \\
\hline
PDE (steady) &
Classify; discretize in space; build sparse linear system. \\
\hline
PDE (time) &
Use method of lines or explicit scheme; compute CFL; choose upwinding if needed. \\
\hline
DAE &
Write semi-explicit/mass form; estimate index; enforce consistent init. \\
\hline
\end{tabular}

\noindent\hrule
\vspace{2pt}

%%%%%%%%%%%%%%%%%%%%%%%%%%%%%%%%%%%%%%%%%%%%%%%%%%%%
\section*{12. Useful MATLAB functions}
\noindent\hrule
\vspace{1pt}
%%%%%%%%%%%%%%%%%%%%%%%%%%%%%%%%%%%%%%%%%%%%%%%%%%%%

Linear algebra:
\begin{itemize}[leftmargin=*]
\item \verb|A\b|: solve \(Ax=b\).
\item \verb|eig(A)|: eigenvalues/vectors.
\item \verb|svd(A)|: SVD, rank, cond.
\item \verb|norm(A,p)|, \verb|cond(A)|.
\end{itemize}

Optimization / SNE:
\begin{itemize}[leftmargin=*]
\item \verb|fsolve|: nonlinear systems.
\item \verb|fminunc|: unconstrained min.
\item \verb|fmincon|: constrained min.
\end{itemize}

Probability / stats:
\begin{itemize}[leftmargin=*]
\item \verb|rand|, \verb|randn|.
\item \verb|binornd|, \verb|poissrnd|.
\item \verb|normpdf|, \verb|normcdf|.
\end{itemize}

ODEs / DAEs:
\begin{itemize}[leftmargin=*]
\item \verb|ode45|: nonstiff IVPs.
\item \verb|ode23s|, \verb|ode15s|: stiff IVPs.
\item \verb|ode15i|: implicit ODE/index-1 DAE.
\end{itemize}

Sparse / linear systems:
\begin{itemize}[leftmargin=*]
\item \verb|sparse|: create sparse matrices.
\item \verb|gmres|, \verb|bicgstab|: iterative solvers.
\end{itemize}



%%%%%%%%%%%%%%%%%%%%%%%%%%%%%%%%%%%%%%%%%%%%%%%%%%%%
\section*{13. Exam patterns \& tricks}
\noindent\hrule
\vspace{1pt}
%%%%%%%%%%%%%%%%%%%%%%%%%%%%%%%%%%%%%%%%%%%%%%%%%%%%

\subsection*{Linear algebra \& discretization}

\begin{itemize}[leftmargin=*]
\item Always mention conditioning: ill-conditioned matrices amplify data/round-off error; well-conditioned ones do not, regardless of solver. Tridiagonal/banded systems from 1D FDM/FVM solve in \(\mathcal{O}(N)\) vs.\ \(\mathcal{O}(N^3)\) for dense. 
\item For “does a solution exist / is it unique?”, map to four subspaces: existence \(\Leftrightarrow b\in\text{range}(A)\); uniqueness \(\Leftrightarrow \text{null}(A)=\{0\}\) (full column rank). 
\end{itemize}

\subsection*{ODE IVPs}

\begin{itemize}[leftmargin=*]
\item For stability questions, immediately apply the test problem \(\dot{x}=\lambda x\) and write \(x_{n+1}=R(z)x_n\), \(z=\lambda\Delta t\). Stability region is \(\{z:|R(z)|\le1\}\). Forward Euler: \(R(z)=1+z\); Backward Euler: \(R(z)=1/(1-z)\). 
\item For accuracy, say: explicit/implicit Euler are first-order in \(\Delta t\); midpoint / trapezoidal are second-order; RK4 is fourth-order. Global error order is one less than the local truncation error order. 
\item Stiffness: if eigenvalues span many orders of magnitude in negative real part, explicit methods need \(\Delta t\ll1/|\lambda_{\max}|\) purely for stability; implicit methods permit much larger \(\Delta t\) for the same problem. 
\end{itemize}

\subsection*{ODE BVPs}

\begin{itemize}[leftmargin=*]
\item Shooting method: define the scalar residual \(h(c)\) (e.g.\ \(h(c)=y(b;c)-y_b\)), then view it as a 1D root-finding problem. Newton in \(c\) typically needs \(h'(c)\) via sensitivities or finite differences. 
\item Finite differences: interior points use centered stencils giving \(\mathcal{O}(\Delta x^2)\) truncation error; if your Taylor expansion yields \(\mathcal{O}(\Delta x)\) you almost certainly misapplied the stencil or indices. 
\item Think of linear BVPs as sparse linear systems: interior nodes give a tridiagonal (1D) or banded (2D) matrix; mention using backslash on sparse matrices for \(\mathcal{O}(N)\)–\(\mathcal{O}(N\,\text{band}^2)\) cost. 
\end{itemize}

\subsection*{PDEs}

\begin{itemize}[leftmargin=*]
\item Classification: look only at second-derivative coefficients \(A,B,C\) in \(A u_{xx}+B u_{xy}+C u_{yy}+\dots=0\). If \(B^2-4AC<0\) elliptic, \(=0\) parabolic, \(>0\) hyperbolic. Lower-order terms do not affect type. 
\item FDM/FVM: “balance form” answers are valued. Always write: accumulation = in-flux \(-\) out-flux \(+\) reaction; then insert 2-point fluxes. This is often enough for most of the credit even if algebra gets messy. 
\item Method of lines: say “discretize space only, get large ODE-IVP system in time, then choose explicit/implicit ODE solver based on stiffness (e.g.\ ode45 vs.\ ode15s).” Mention this pipeline explicitly. 
\item Von Neumann / CFL: plug Fourier mode \(e^{ikx}\) into the scheme to get growth factor \(G(k)\); require \(|G(k)|\le1\) for all \(k\). For upwind advection, recover \(0\le\text{CFL}\le1\); central advection + FE is typically unstable. 
\end{itemize}

\subsection*{DAEs}

\begin{itemize}[leftmargin=*]
\item Index trick: for semi-explicit DAEs \(x'=f(t,x,y),\ 0=g(t,x,y)\), differentiate algebraic equations until you can solve for all derivatives; the number of differentiations is the index (ODE-IVPs are index-0). 
\item Consistent initialization: check both explicit algebraic constraints and any hidden ones revealed by differentiation at \(t_0\). If \((x_0,y_0)\) does not satisfy all, the ICs are inconsistent. 
\item Mass-matrix form: if \(M\dot{z}=F(t,z)\) with singular \(M\), say “this is a DAE; use an implicit stiff solver (e.g.\ ode15s / ode15i) and supply a consistent initial condition, possibly computed with a ‘decic’-style routine.” 
\end{itemize}

\subsection*{Probability, estimation, MCMC}

\begin{itemize}[leftmargin=*]
\item Likelihood vs.\ prior vs.\ posterior: exams love “write the likelihood” and “derive MLE / MAP” for simple models (normal, Poisson, exponential). Always write posterior \(\propto\) likelihood \(\times\) prior and normalize only if asked. 
\item For linear regression / experimental design, tie “better experiment” answers to the design matrix \(X\): more spread in inputs and less collinearity \(\Rightarrow\) more informative (smaller parameter covariance). 
\item MCMC “integral form”: the standard answer is \(\displaystyle \frac{\int g(x)f(x)\,dx}{\int f(x)\,dx}\). Recognize code patterns where samples are used to estimate expectations of this form; mention unbiasedness as \(M\to\infty\). 
\end{itemize}

\noindent\hrule
\vspace{2pt}


%%%%%%%%%%%%%%%%%%%%%%%%%%%%%%%%%%%%%%%%%%%%%%%%%%%%
\section*{14. Taylor Expansion}
\noindent\hrule
\vspace{1pt}
%%%%%%%%%%%%%%%%%%%%%%%%%%%%%%%%%%%%%%%%%%%%%%%%%%%%
A Taylor expansion of a sufficiently smooth scalar function \(f\) about a point \(x_0\) is
\[
f(x) \approx \sum_{k=0}^{n} \frac{f^{(k)}(x_0)}{k!} (x - x_0)^k.
\]

For specific orders:
\[
\text{Order 1: } f(x) \approx f(x_0) + f'(x_0)(x - x_0),
\]
\[
\text{Order 2: } f(x) \approx f(x_0) + f'(x_0)(x - x_0)
    + \frac{1}{2} f''(x_0)(x - x_0)^2,
\]
\[
\text{Order 3: } f(x) \approx f(x_0) + f'(x_0)(x - x_0)
    + \frac{1}{2} f''(x_0)(x - x_0)^2 +...
\]
\[
    ...+ \frac{1}{6} f^{(3)}(x_0)(x - x_0)^3.
\]


\end{multicols}



\pagebreak

\noindent\hrule
\vspace{2pt}

\section*{Finite difference coefficients}
\noindent\hrule
\vspace{1pt}

\subsection*{Central finite differences (uniform $h$)}

\small
\begin{center}
\begin{tabular}{c|c|rrrrrrrrrrr}
Deriv. & Acc. & $-5$ & $-4$ & $-3$ & $-2$ & $-1$ & $0$ & $1$ & $2$ & $3$ & $4$ & $5$ \\
\hline
1 & 2 &   &   &   &   & $-1/2$ & 0 & $1/2$ &   &   &   &   \\
1 & 4 &   &   &   & $1/12$ & $-2/3$ & 0 & $2/3$ & $-1/12$ &   &   &   \\
1 & 6 &   &   & $-1/60$ & $3/20$ & $-3/4$ & 0 & $3/4$ & $-3/20$ & $1/60$ &   &   \\
1 & 8 &   & $1/280$ & $-4/105$ & $1/5$ & $-4/5$ & 0 & $4/5$ & $-1/5$ & $4/105$ & $-1/280$ &   \\
2 & 2 &   &   &   &   & 1 & $-2$ & 1 &   &   &   &   \\
2 & 4 &   &   &   & $-1/12$ & $4/3$ & $-5/2$ & $4/3$ & $-1/12$ &   &   &   \\
2 & 6 &   &   & $1/90$ & $-3/20$ & $3/2$ & $-49/18$ & $3/2$ & $-3/20$ & $1/90$ &   &   \\
2 & 8 &   & $-1/560$ & $8/315$ & $-1/5$ & $8/5$ & $-205/72$ & $8/5$ & $-1/5$ & $8/315$ & $-1/560$ &   \\
3 & 2 &   &   &   & $-1/2$ & 1 & 0 & $-1$ & $1/2$ &   &   &   \\
3 & 4 &   &   & $1/8$ & $-1$ & $13/8$ & 0 & $-13/8$ & 1 & $-1/8$ &   &   \\
3 & 6 &   & $-7/240$ & $3/10$ & $-169/120$ & $61/30$ & 0 & $-61/30$ & $169/120$ & $-3/10$ & $7/240$ &   \\
4 & 2 &   &   &   & 1 & $-4$ & 6 & $-4$ & 1 &   &   &   \\
4 & 4 &   &   & $-1/6$ & 2 & $-13/2$ & $28/3$ & $-13/2$ & 2 & $-1/6$ &   &   \\
4 & 6 &   & $7/240$ & $-2/5$ & $169/60$ & $-122/15$ & $91/8$ & $-122/15$ & $169/60$ & $-2/5$ & $7/240$ &   \\
5 & 2 &   &   & $-1/2$ & 2 & $-5/2$ & 0 & $5/2$ & $-2$ & $1/2$ &   &   \\
5 & 4 &   & $1/6$ & $-3/2$ & $13/3$ & $-29/6$ & 0 & $29/6$ & $-13/3$ & $3/2$ & $-1/6$ &   \\
5 & 6 & $-13/288$ & $19/36$ & $-87/32$ & $13/2$ & $-323/48$ & 0 & $323/48$ & $-13/2$ & $87/32$ & $-19/36$ & $13/288$ \\
6 & 2 &   &   & 1 & $-6$ & 15 & $-20$ & 15 & $-6$ & 1 &   &   \\
6 & 4 &   & $-1/4$ & 3 & $-13$ & 29 & $-75/2$ & 29 & $-13$ & 3 & $-1/4$ &   \\
6 & 6 & $13/240$ & $-19/24$ & $87/16$ & $-39/2$ & $323/8$ & $-1023/20$ & $323/8$ & $-39/2$ & $87/16$ & $-19/24$ & $13/240$ \\
\end{tabular}
\end{center}
\normalsize

\subsection*{Forward finite differences (uniform $h$)}

\small
\begin{center}
\begin{tabular}{c|c|rrrrrrrrr}
Deriv. & Acc. & 0 & 1 & 2 & 3 & 4 & 5 & 6 & 7 & 8 \\
\hline
1 & 1 & $-1$ & 1 &   &   &   &   &   &   &   \\
1 & 2 & $-3/2$ & 2 & $-1/2$ &   &   &   &   &   &   \\
1 & 3 & $-11/6$ & 3 & $-3/2$ & $1/3$ &   &   &   &   &   \\
1 & 4 & $-25/12$ & 4 & $-3$ & $4/3$ & $-1/4$ &   &   &   &   \\
1 & 5 & $-137/60$ & 5 & $-5$ & $10/3$ & $-5/4$ & $1/5$ &   &   &   \\
1 & 6 & $-49/20$ & 6 & $-15/2$ & $20/3$ & $-15/4$ & $6/5$ & $-1/6$ &   &   \\
2 & 1 & 1 & $-2$ & 1 &   &   &   &   &   &   \\
2 & 2 & 2 & $-5$ & 4 & $-1$ &   &   &   &   &   \\
2 & 3 & $35/12$ & $-26/3$ & $19/2$ & $-14/3$ & $11/12$ &   &   &   &   \\
2 & 4 & $15/4$ & $-77/6$ & $107/6$ & $-13$ & $61/12$ & $-5/6$ &   &   &   \\
2 & 5 & $203/45$ & $-87/5$ & $117/4$ & $-254/9$ & $33/2$ & $-27/5$ & $137/180$ &   &   \\
2 & 6 & $469/90$ & $-223/10$ & $879/20$ & $-949/18$ & 41 & $-201/10$ & $1019/180$ & $-7/10$ &   \\
3 & 1 & $-1$ & 3 & $-3$ & 1 &   &   &   &   &   \\
3 & 2 & $-5/2$ & 9 & $-12$ & 7 & $-3/2$ &   &   &   &   \\
3 & 3 & $-17/4$ & $71/4$ & $-59/2$ & $49/2$ & $-41/4$ & $7/4$ &   &   &   \\
3 & 4 & $-49/8$ & 29 & $-461/8$ & 62 & $-307/8$ & 13 & $-15/8$ &   &   \\
3 & 5 & $-967/120$ & $638/15$ & $-3929/40$ & $389/3$ & $-2545/24$ & $268/5$ & $-1849/120$ & $29/15$ &   \\
3 & 6 & $-801/80$ & $349/6$ & $-18353/120$ & $2391/10$ & $-1457/6$ & $4891/30$ & $-561/8$ & $527/30$ & $-469/240$ \\
4 & 1 & 1 & $-4$ & 6 & $-4$ & 1 &   &   &   &   \\
4 & 2 & 3 & $-14$ & 26 & $-24$ & 11 & $-2$ &   &   &   \\
4 & 3 & $35/6$ & $-31$ & $137/2$ & $-242/3$ & $107/2$ & $-19$ & $17/6$ &   &   \\
4 & 4 & $28/3$ & $-111/2$ & 142 & $-1219/6$ & 176 & $-185/2$ & $82/3$ & $-7/2$ &   \\
4 & 5 & $1069/80$ & $-1316/15$ & $15289/60$ & $-2144/5$ & $10993/24$ & $-4772/15$ & $2803/20$ & $-536/15$ & $967/240$ \\
\end{tabular}
\end{center}
\normalsize

\subsection*{Backward finite differences (uniform $h$)}

\small
\begin{center}
\begin{tabular}{c|c|rrrrrrrrr}
Deriv. & Acc. & $-8$ & $-7$ & $-6$ & $-5$ & $-4$ & $-3$ & $-2$ & $-1$ & 0 \\
\hline
1 & 1 &   &   &   &   &   &   &   & $-1$ & 1 \\
1 & 2 &   &   &   &   &   &   & $1/2$ & $-2$ & $3/2$ \\
1 & 3 &   &   &   &   &   & $-1/3$ & $3/2$ & $-3$ & $11/6$ \\
2 & 1 &   &   &   &   &   &   & 1 & $-2$ & 1 \\
2 & 2 &   &   &   &   &   & $-1$ & 4 & $-5$ & 2 \\
3 & 1 &   &   &   &   &   & $-1$ & 3 & $-3$ & 1 \\
3 & 2 &   &   &   &   & $3/2$ & $-7$ & 12 & $-9$ & $5/2$ \\
4 & 1 &   &   &   &   & 1 & $-4$ & 6 & $-4$ & 1 \\
4 & 2 &   &   &   & $-2$ & 11 & $-24$ & 26 & $-14$ & 3 \\
\end{tabular}
\end{center}
\normalsize



\end{document}
