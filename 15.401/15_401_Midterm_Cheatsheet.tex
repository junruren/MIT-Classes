\documentclass[10pt,landscape]{article}
\usepackage{multicol}
\usepackage{calc}
\usepackage{ifthen}
\usepackage[landscape]{geometry}
\usepackage{listings}
\usepackage{amsmath,amsthm,amsfonts,amssymb}
\usepackage{mathtools}
\usepackage{color,graphicx,overpic}
\usepackage{hyperref}
\usepackage[dvipsnames]{xcolor}
\usepackage{transparent}

\usepackage{MnSymbol}
\usepackage{graphicx}
\usepackage{wrapfig}
\usepackage{tikz}

\usepackage{blindtext}

% This sets page margins to .1 inch if using letter paper, and to 1cm
% if using A4 paper. (This probably isn't strictly necessary.)
% If using another size paper, use default 1cm margins.
\ifthenelse{\lengthtest { \paperwidth = 11in}}
    { \geometry{top=0.2in,left=0.2in,right=0.2in,bottom=0.2in} }
    {\ifthenelse{ \lengthtest{ \paperwidth = 297mm}}
        {\geometry{top=1cm,left=1cm,right=1cm,bottom=1cm} }
        {\geometry{top=1cm,left=1cm,right=1cm,bottom=1cm} }
    }

% Turn off header and footer
\pagestyle{empty}

% Redefine section commands to use less space
\makeatletter
\renewcommand{\section}{\@startsection{section}{1}{0mm}%
                                {-1ex plus -.5ex minus -.2ex}%
                                {0.5ex plus .2ex}%x
                                {\normalfont\large\bfseries}}
\renewcommand{\subsection}{\@startsection{subsection}{2}{0mm}%
                                {-1ex plus -.5ex minus -.2ex}%
                                {0.5ex plus .2ex}%
                                {\normalfont\normalsize\bfseries}}
\renewcommand{\subsubsection}{\@startsection{subsubsection}{3}{0mm}%
                                {-1ex plus -.5ex minus -.2ex}%
                                {0.5ex plus .2ex}%
                                {\normalfont\footnotesize\bfseries}}
\makeatother

% Itemize to use less space
\usepackage{enumitem}
\setlist{leftmargin=*, nosep}
\setenumerate{nosep}

% Define BibTeX command
\def\BibTeX{{\rm B\kern-.05em{\sc i\kern-.025em b}\kern-.08em
    T\kern-.1667em\lower.7ex\hbox{E}\kern-.125emX}}

% Don't print section numbers
\setcounter{secnumdepth}{0}


\setlength{\parindent}{0pt}
\setlength{\parskip}{0pt plus 0.5ex}

%My Environments
\newtheorem{example}[section]{Example}

\newcommand{\Blue}[1]{\noindent{\textcolor{Blue}{\textbf{#1}}}:}
\newcommand{\Red}[1]{\noindent{\textcolor{BrickRed}{\textbf{#1}}}:}
\newcommand{\Green}[1]{\noindent{\textcolor{PineGreen}{\textbf{#1}}}:}
\newcommand{\Hint}[1]{\noindent{\textcolor{Orange}{#1}}}

\newcommand*{\eg}{e.g.\@\xspace}
\newcommand*{\ie}{i.e.\@\xspace}
\newcommand*{\Eg}{E.g.\@\xspace}
\newcommand*{\Ie}{I.e.\@\xspace}
\newcommand*{\esp}{esp.\@\xspace}
\newcommand*{\wrt}{\ifmmode \stext{w.r.t.} \else w.r.t.\@\xspace \fi}


\usepackage{draftwatermark}
% Configure the watermark
\SetWatermarkText{Made with love by \texttt{junruren}} % Set the watermark text
\SetWatermarkScale{2.5}            % Adjust the scale of the watermark
\SetWatermarkLightness{0.9}      % Set the lightness (closer to 1 is more faded)
% -----------------------------------------------------------------------

\begin{document}
\raggedright
\scriptsize

\begin{multicols}{4}
% multicol parameters
% These lengths are set only within the two main columns
%\setlength{\columnseprule}{0.25pt}
\setlength{\premulticols}{1pt}
\setlength{\postmulticols}{1pt}
\setlength{\multicolsep}{1pt}
\setlength{\columnsep}{2pt}
\subsection{Net Present Value (NPV)}
\Blue{Discoint rate $r$} you're indifferent between receiving \$1 today and $\$\frac{1}{1+r}$ in one period.

\Blue{Present Value (PV)} \fbox{$PV(CF_t) = \frac{CF_t}{(1+r)^t}$}
how much a cash flow (CF) at time $t$ is worth at time 0 (today). Computing a PV is often called \Hint{``discounting''}.

\Red{NPV} \fbox{$NPV = \sum_{t=0}^{T} \frac{CF_t}{(1+r)^t}$} summs over PVs of all cash flows in a project.
\begin{itemize}
    \item \Green{Scalability} $NPV(\alpha CF_1, \dots, \alpha CF_T) = \alpha NPV(CF_1, \dots, CF_T)$
    \item \Green{Additivity} $NPV(X_1 + Y_1, \dots, X_T + Y_T) = NPV(X_1, \dots, X_T) + NPV(Y_1, \dots, Y_T)$
    \item \Green{Breaking up by time} $NPV(CF_1, \dots, CF_T) = NPV(CF_1, \dots, CF_j) + NPV(CF_{j+1}, \dots, CF_T)$
\end{itemize}

\Blue{Future Value (FV)}
\fbox{{$FV_T(CF_0) = CF_0(1+r)^T$}}
how much a cash flow at time 0 (today) is worth in $T$ periods.

\Blue{Perpetuity}
\begin{itemize}
    \item \underline{Constant} recurring cash flow $A$ forever starting \textbf{1 period from now}: \fbox{$PV = \frac{A}{r}$}
    \item \underline{Growing} perpetuity starting \textbf{1 period from now} with cash flow $A$, growth rate $g$: \fbox{$PV = \frac{A}{r-g} (r > g)$}
\end{itemize}

\Blue{Annuity}
\begin{itemize}
    \item \underline{Constant} recurring cash flow $A$ for $T$ periods starting \textbf{1 period from now}:
        \Hint{(\Eg a loan)}
        \fbox{$PV = \frac{A}{r} \left(1 - \frac{1}{(1+r)^T}\right)$}
        \fbox{$FV = PV \cdot (1+r)^T = A \frac{(1+r)^T - 1}{r}$}
    \item \underline{Growing} annuity starting \textbf{1 period from now} with cash flow $A$, growth rate $g$ for $T$ periods.
        \begin{itemize}
            \item If $r \neq g$
                \fbox{$PV = \frac{A}{r-g} \left(1 - \frac{(1+g)^T}{(1+r)^T}\right)$};
                \fbox{$FV = A \left(\frac{(1+r)^T - (1+g)^T}{r-g}\right)$}
            \item If $r = g$
                \fbox{$PV = T \left(\frac{A}{1+r}\right)$};
                \fbox{$FV = T \cdot A \cdot (1+r)^{T-1}$}
        \end{itemize}
\end{itemize}

\end{multicols}
\end{document}